\documentclass[french]{article}

% Packages for french documents
\usepackage{babel}
\usepackage[utf8]{inputenc}
\usepackage[T1]{fontenc}

% Define some colors
\usepackage{color}
\definecolor{string}{RGB}{100, 200, 0}
\definecolor{comment}{RGB}{150, 150, 150}
\definecolor{identifier}{RGB}{100, 100, 200}

% Source code style
\usepackage{listings}
\lstset{
	basicstyle=\ttfamily,         % sets font style for the code
	frame=single,                 % adds a frame around the code
	showstringspaces=false,       % underline spaces within strings
	tabsize=4,                    % sets default tabsize to 2 spaces
	breaklines=true,              % sets automatic line breaking
	breakatwhitespace=true,       % sets if automatic breaks should only happen at whitespace
	keywordstyle=\color{magenta}, % sets color for keywords
	stringstyle=\color{string},   % sets color for strings
	commentstyle=\color{comment}, % sets color for comments
	emphstyle=\color{identifier}, % sets color for comments
}

% Hyperlinks
\usepackage[hyphens]{url}
\usepackage[hidelinks]{hyperref}

\title{Projet ARA : Paxos}
\date{Février 2021}
\author{Sylvain Joube, Nicolas Peugnet}

\begin{document}

\maketitle

\tableofcontents

\section{Introduction}

Le but de ce projet est de d'étudier l'efficacité de l'algorithme de Paxos en faisant varier un certain nombre de paramètres.
La première chose à faire a été de coder l'algorithme à l'aide d'un simulateur afin de pouvoir lancer plusieurs expérimentations.
Nous avons utilisé PPI\footnote{\url{https://github.com/PolyProcessInterface/ppi}} pour cette tâche,
car cette API permet de lancer le code soit sur le simulateur Peersim soit via MPI et parce que réutiliser un projet réalisé par le passé était plus motivant pour au moins l'un de nous deux.
Ce n'était pas une si bonne idée que ça.

\section{Étude expérimentale 1}

Le but de cette première étude expérimentale est d'évaluer l'efficacité d'une unique itération de l'algorithme.
Les valeurs prises en compte pour quantifier cette efficacité sont :

\begin{enumerate}
	\item Le \textbf{temps de convergence} : le temps nécessaire au protocole pour que tous les nœuds reconnaissent le même leader.
	\item Le \textbf{nombre de rounds} nécessaires pour atteindre une majorité, c’est-à-dire pour que le consensus soit atteint
	\item Le \textbf{nombre de messages} émis
\end{enumerate}













\section{Démo \LaTeX}

Tout d'abord faisons un paragraphe, pour ça il suffit de sauter 2 lignes.
1 seul saut de ligne équivaut à un espace (comme en md).
Il est donc \emph{préférable} de sauter une ligne à chaque phrase.
De cette manière \lstinline{git} peut plus efficacement suivre les modifications.
On remarque aussi que \LaTeX ajoute tout seul les espaces insécables avant les double ponctuations en mode FR, ce qui est cool !

Il est possible de mettre un mot en \emph{emphase}, on voit dans l'exemple suivant en quoi c'est intéressant.
\textit{Dans cette phrase, un mot est \emph{particulièrement} important}.
Oui et on peut mettre du \textbf{texte en \emph{gras} aussi.}


Mais bon en vrai il y a quand même des trucs un peu relou.
Style le fait de devoir mettre un \lstinline{\} devant une bonne partie des caractères spéciaux.
Voilà un petit tableau (incomplet) des caractères réservés qu'il faut \emph{escaper} :

\begin{tabular}{ c l }
	Caractère & Escaping \\
	\hline
	\{ & \verb|\{| \\
	\} & \verb|\}| \\
	\% & \verb|\%| \\
	\$ & \verb|\$| \\
	\& & \verb|\&| \\
	\# & \verb|\#| \\
	\_ & \verb|\_| \\
	\textbackslash & \verb|\textbackslash|
\end{tabular}

\newpage
Un petit bloc de code parce que quand même, on est des développeurs quoi.
\begin{lstlisting}[language=java]
int v = 42;
// The main
void main() {
	System.out.println("hello world!");
}
\end{lstlisting}

Et tant qu'on y est, une petite liste\footnote{Mais en vrai le meilleur truc c'est vraiment les footnotes} :
\begin{itemize}
	\item élément 1
	\item élément 2
\end{itemize}

Pour vraiment apprendre comment fonctionne \LaTeX j'ai trouvé un tutoriel
\footnote{\raggedright\url{https://zestedesavoir.com/tutoriels/826/introduction-a-latex/1319_creer-vos-premiers-documents/5241_premiers-codes/}}
vraiment sympa, mais je ne pense pas qu'on ait le temps haha.

\end{document}
